
\chapter{Einleitung}

\section{Motivation}
Der Begriff \enquote{autonomes Fahren} hat spätestens seit den Tesla Autos einen allgemeinen Bekanntheitsgrad erreicht. Das dabei allerdings nur ein erweitertes Abstand- und Spurhaltesystem
für die Autobahn gemeint ist, wird oft unterschlagen.

Die Umsetzung von autonomen Fahren in Städten wird von Autoherstellern aber bereits stark erforscht.
Mit einer Möglichkeit einer Umsetzung von autonomen Fahren in eine Kreuzung wird in dieser Arbeit erläutert.
Zusammen mit vielen anderen Themengebieten kann somit das autonome Fahren ermöglicht werden.
\todo{fix 404}

\section{Projektkontext}
\todo{mec-view.de}

\section{Zielsetzung}

Das Ziel ist es, eine alternative Implementierung des MEC-View Servers in Rust zu schaffen.
Durch die Garantien \todo{ref} von Rust wird erhofft, dass der menschliche Faktor als Fehlerquelle gemindert wird und somit eine fehlertolerantere und sicherere Implementation geschaffen wird.
\todo{möglichst Beibehalt der Architektur?}


\section{Aufbau der Arbeit}

Diese Arbeit ist im wesentlichen in die folgenden Themengebiete aufgeteilt: Grundlagen, Anforderungs- und Systemanalyse, Systementwurf und Implementation und Auswertung.

Im Themengebiet Grundlagen sollen wesentliche Bestandteile dieser Arbeit erläutert und erklärt werden.
Hierzu zählt zum einen die Programmiersprache Rust in ihrer Entstehungsgeschichte \todo{ref}, Garantien \todo{ref}  und Sprachfeatures \todo{ref}, zum anderen die hochperformante, serverbasierte Kommunikationsplattform mit ihren Protokollen \todo{ref} und dem Systemkontext in dem diese betrieben wird.

In der Anforderungs- und Systemanalyse wird der Kontext in dem das System betrieben werden soll genauer betrachtet. Umzusetzende funktionale und nicht-funktionale Anforderungen werden aufgestellt sowie eine Übersicht von Systemen mit denen interagiert wird.

Das Themengebiet Systementwurf und Implementation befasst sich mit dem theoretischen und praktischen Lösen der im vorherigen Kapitel aufgestellten Anforderungen. Aufgrund der Tatsache, dass es sich hierbei
um eine alternative Implementation handelt, wird zur bestehenden C++ Implementation Bezug genommen.
Auf architektonische Unterschiede im Systementwurf, die sich aufgrund von Sprach- und Bibliotheksunterschiede, werden hier genauer beschrieben.

Zuletzt wird eine Auswertung der Implementation aufgezeigt. \todo{michael.write\_more();}
\chapter*{Kurzfassung}


Autonome Fahrzeuge sind Bestandteil vieler Zukunftsvisionen und werden von der Industrie bereits erforscht und entwickelt.
Hierfür nötige und komplexe Software bietet eine höhere Wahrscheinlichkeit für Fehler, die sogar Menschenleben kosten könnten.
%Weil Menschenleben gefährdet werden können, sind unbehandelte Fehlersituationen oder undefiniertes Verhalten gefürchtete Fehler.
Konventionell eingesetzte Programmiersprachen wie C++ weisen eine hohe Ausführgeschwindigkeit auf, bieten durch naive Herangehensweisen aber auch Potenzial für diese fatalen Laufzeitfehler.

Die Programmiersprache Rust soll eine hohe Ausführgeschwindigkeit erreichen, dank ihrer Garantien dieses Fehlerpotenzial jedoch vermeiden.
Ob Rust als Alternative im echtzeitnahen Umfeld geeignet ist, wird in dieser Bachelorarbeit durch den Entwurf und die Implementation einer Kommunikationsplattform überprüft.
Diese Implementation wird mit einem bestehenden C++ Prototypen im Laufzeitverhalten verglichen.

%Im Rahmen dieser Bachelorarbeit wird untersucht, ob die Programmiersprache Rust als Alternative für die Programmiersprache C++ im echtzeitnahen Umfeld geeignet ist.

%Im Rahmen dieser Bachelorarbeit wird untersucht, ob die Programmiersprache Rust als Alternative für die Programmiersprache C++ im echtzeitnahen Umfeld geeignet ist.
%Programmiersprachen wie C++ weisen eine hohe Ausführgeschwindigkeit auf, bieten aber auch ein Potenzial für fatale Laufzeitfehler und undefiniertes Verhalten.
%Beim autonomen Fahren kann eine unbehandelte Fehlersituation oder ein undefiniertes Verhalten Menschenleben kosten.
%Durch die Garantien von Rust soll eine hohe Ausführgeschwindigkeit erreicht werden, ohne dieses Fehlerpotential zu bieten.

Im Grundlagenkapitel werden  die Eigenschaften, Versprechen und Garantien von Rust untersucht.
Zudem wird ermittelt, welche Eigenschaften erfüllt sein müssen, um eine hochperformante Kommunikationsplattform umzusetzen.

In den Kapiteln \enquote{Implementierung} und \enquote{Auswertung} werden Vorgehensweisen, Schwierigkeiten bei der Implementierung und das Ergebnis dieser Arbeit präsentiert.

%\textbf{Schlagworte:} Rust, FFI, ASN.1, MEC, sichere Nebenläufigkeit, asynchrone Kommunikation
%\textbf{Schlagworte: Rust, FFI, ASN.1, MEC, sichere Nebenläufigkeit, asynchrone Ein- und Ausgabe}
\paragraph{Schlagworte:} Rust, FFI, ASN.1, sichere Nebenläufigkeit, asynchrone Ein- und Ausgabe
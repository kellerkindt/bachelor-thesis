
\chapter{Implementierung}

In diesem Kapitel wird auf Besonderheiten bei der Implementierung eingegangen.


%Weitestgehend wie in \autoref{draft:architecture}, aber Futures, tokio, Queues

\todo{channel architektur pattern \cite[167]{douglass2003real}?, diagramm message decode steps}

\todo{proxy communication}

\section{Bindgen für ASN}

\label{impl:issue:ffi}
\todo{schnell problem: kein asn->rs compiler, c bindings aufwending -> autogen via bindgen}

\todo{link issues fixed in commit d5d694c}

\todo{impl Drop / free structs, important but do not overengineer}


\subsection{Continuous Integration mittels Jenkins}

\todo{screenshot + explain}

\todo{prüft rustfmt, wertet test-coverage aus und führt tests aus --> schnelle rückmeldung bei fehlern}

\todo{tdd schwierig weil viel integration mit framework / async, schnell wird daraus integrationstest(?, unerwünscht)}

\todo{rust-clippy: https://github.com/rust-lang-nursery/rust-clippy}


\section{Strategien für Performance}

\todo{\enquote{bypass} for algorithm->>many vehicles}





\todo{diagramm: jede future ein eigenes "paket", queues dazwischen für kommunikation}

\todo{catch panic? mention how in \autoref{rust:no_null}}

\todo{SRP, impl Object separat from impl CommandProcessor for Object, testability?}

\subsection{Tokio}

	\subsection{Generalisierung mittels Aufzählung für nicht erweiterbare Anzahl von Elementen}
	
	\subsection{libmessages make unsafe libmessages-sys safe}
	
	\subsection{Vorgehen, bindgen tests, C/unsafe/wrapper -> nach sicher ->  architektur entwickeln}
	

	
	\subsection{Unerwartete Schwierigkeiten}
	
	\todo{Trailing zeroes issue, commit a02496d + tagged, libmessages/src/asn.rs:17 bzw :31 bzw :32 Ok((result.encoded as usize + 7) / 8}
	
	\todo{unterschiedlicher Heap libc / rust}
	
	
	\subsection{--help}

\todo{bei der entwicklung nie segfault/deadlock gehabt, nur einmal heap problem wegen libc}

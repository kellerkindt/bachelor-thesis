
\newglossaryentry{ffi}{
	name=Foreign Function Interface,
	description={
		Beschreibt den Mechanismus wie ein Programm das in einer Programmiersprache geschrieben ist,
		Funktionen aufrufen kann, die einer einer anderen Programmiersprache geschrieben wurden. \cite{wiki:ffi}
	}
}

\newglossaryentry{llvm}{
	name={LLVM},
	description={
		Früher \enquote{Low Level Virtual Machine} \cite{wiki:llvm}, heute Eigenname;
		ist eine \enquote{Ansammlung von modularen und wiederverwendbaren Compiler- und Werkzeugtechnologien} \cite{llvm:home}.
		Unterstützt eine große Anzahl von Zielplattformen, u.a. X86, X86-64, PowerPC, PowerPC-64, ARM, Thumb, ... \cite{llvm:features}. \\
		Bei der Compilierung wird der Programmcode zuerst in einen Assembler-ähnlichen Code übersetzt, der daraufhin von LLVM zu Maschinencode der Zielplattform compiliert und optimiert wird.
		% Somit kann auf ein bereits großes Spektrum an Optimierungen zurückgegriffen werden.
	}
}

\newglossaryentry{github}{
	name={GitHub},
	description={
		Plattform zum Hosten von \gls{git}-Repositories inklusive eingebautem Issue-Tracker und Wiki.
		Änderungen an Quellcode können vorgeschlagen und durch die Projektverantwortlichen übernommen werden. Bietet auch die Möglichkeit, eine kontinuierliche Integrationssoftware einzubinden, um automatisierte Tests auf momentanen Quellcode und auch für Änderungen auszuführen. Eine vorgeschlagene Änderung kann somit vor Übernahme auf Kompatibilität überprüft werden.
	}
}

\newglossaryentry{git}{
	name={git},
	description={
		(dt. Blödmann) ist eine Software zur Versionierung von Quelldateien, entwickelt von Linus Torvalds 2005 \cite{wiki:git}.
	}
}


\chapter{Systemanalyse}

\cite[502]{goll2012methoden}


\section{Systemkontextdiagramm}

In der folgenden Abbildung sind die Systemgrenzen aufgezeigt.

\begin{figure}[H]
	\begin{tikzpicture}[node distance=5]
		\node [draw, rectangle, drop shadow, fill=white] (F) {Fahrzeug};
		\node [draw, circle, drop shadow, fill=white, text width=3cm, align=center] (C) [below right=of F] {MEC-Server};
		\node [draw, rectangle, drop shadow, fill=white] (S) [below right=of C] {Sensor};
%		\node [draw, rectangle, drop shadow, fill=white] (A) [above right=of C] {Fusions-Algorithmus};
		
		\node (ctrl1) [above=of S] {};
		\node (ctrl2) [above right=of C] {};
		\node (ctrl3) [right=of C] {};
		\node (ctrl4) [below=of F] {};
		\node (ctrl5) [left=of C] {};
		\node (ctrl6) [below=of ctrl5] {};
		\node (ctrl7) [above=of C] {};
		\node (ctrl8) [right=of F] {};
		
		%
		% Fahrzeug -> MEC-Server
		%
		\path[draw, -{Latex[scale=2.0]}, dashed] (F)
				edge [bend right] node [fill=white] {deabonnieren} (C)
				edge [bend left]  node [fill=white] {abonnieren} (C)
				.. controls (ctrl7) and (ctrl8) ..  node [fill=white] {Als Fahrzeug registrieren} (C);
				
		%
		% MEC-Server -> Fahrzeug
		%
		\path[draw, -{Latex[scale=2.0]}] (C)
				edge node [fill=white] {EnvironmentFrame} (F)
				.. controls (ctrl4) and (ctrl5) ..  node [fill=white] {Sektoren} (F);
				
		%
		% MEC-Server -> Algorithmus
		%
%		\path[draw, -{Latex[scale=2.0]}] (C)
%				edge [bend right] node [fill=white] {SensorFrame} (A);
		
		%
		% Algorithmus -> MEC-Server
		%
%		\path[draw, -{Latex[scale=2.0]}] (A)
%				edge [bend right] node [fill=white] {EnvironmentFrame} (C);
		
		%
		% Sensor -> MEC-Server
		%		
		\path[draw, -{Latex[scale=2.0]}, dashed] (S)
			.. controls (ctrl3) and (ctrl1) ..  node [fill=white] {Als Sensor registrieren} (C);
			
		\path[draw, -{Latex[scale=2.0]}] (S)
			edge node [fill=white] {Sensor(Idle)Frame} (C);
				
		%
		% MEC-Server -> Sensor
		%
		\path[draw, -{Latex[scale=2.0]}, dashed] (C)
			edge [bend left]  node [fill=white] {abonnieren} (S)
			edge [bend right] node [fill=white] {deabonnieren} (S);
			
				
				
		
		\path[draw, -{Latex[scale=2.0]}] (4, -8) -- (.5, -8) node [pos=.5, above] {Datenfluss};
		\path[draw, -{Latex[scale=2.0]}, dashed] (4, -9) -- (.5, -9) node [pos=.5, above] {Kontrollfluss};
			
		%\umlinherit{F}{L}
		%\umlinherit{S}{L}
	\end{tikzpicture}
	\centering
	\label{system_context}
	\caption{Systemkontextdiagramm}
\end{figure}


\section{Anwendungsfalldiagramme}
\todo{was wirklich umgesetzt sein wird}

\subsection{MEC-Server}

Das Anwendungsfalldiagramm in \autoref{use_case:car} zeigt die Funktionalität des Servers, die gegenüber einem Fahrzeug un einem Sensor zur Verfügung stellt werden soll.
Darauf folgend sind diese Anwendungsfälle genauer erklärt.

\begin{figure}[H]
	\begin{tikzpicture}[node distance=5]
		\begin{umlsystem}{MEC-Server}
%			\umlusecase[y=4,name=u0]{Verbindung aufbauen}
			\umlusecase[y=2,name=u3]{Als Fahrzeug registrieren}
			\umlusecase[y=0,name=u1]{Umgebungsmodell abonnieren}
			\umlusecase[y=-2,name=u2]{Umgebungsmodell deabonnieren}
			\umlusecase[y=-4,name=u4]{Als Sensor registrieren}
			\umlusecase[y=-6,name=u5]{SensorFrame zusenden}
			\umlusecase[y=-8,name=u6]{SensorIdleFrame zusenden}
%			\umlusecase[y=-4,name=u4]{Umfeldmodel übergeben}
		\end{umlsystem}
		\umlactor[x=-6.5,y=0]{Fahrzeug}
%		\umlactor[x=-6,y=-4]{Fusions-Algorithmus}
		\umlactor[x=6.5,y=-6]{Sensor}
%		\umlassoc{Fahrzeug}{u0}
		\umlassoc{Fahrzeug}{u1}
		\umlassoc{Fahrzeug}{u2}
		\umlassoc{Fahrzeug}{u3}
%		\umlassoc{Sensor}{u0}
%		\umlassoc{Sensor}{u1}
		\umlassoc{Sensor}{u4}
		\umlassoc{Sensor}{u5}
		\umlassoc{Sensor}{u6}
%		\umlassoc{Fusions-Algorithmus}{u4}
	\end{tikzpicture}
	\centering
	\caption{Anwendungsfalldiagramm des MEC-Servers}
	\label{use_case:car}
\end{figure}
\begin{itemize}
	\item \textbf{Als Fahrzeug registrieren [Vorbedingung: noch nicht registriert]} \\
	Ein neu verbundenes Fahrzeug kann sich dem Server gegenüber als Fahrzeug registrieren.
	Eine Registrierung kann für jede Verbindung  nur einmal vorgenommen werden und wird durch die Übermittlung einer \textit{ClientRegistration}-Nachricht durchgeführt (siehe \autoref{msg:client_registration}).
	
	\item \textbf{Umgebungsmodell abonnieren [Vorbedingung: als Fahrzeug registriert]} \\
	Ein Fahrzeug kann das Umgebungsmodell abonnieren, woraufhin neue Modelle vom Server an das Fahrzeug übermittelt werden sollen.
	Ein Abonnement wird durch eine \textit{UpdateSubscription}-Nachricht aktualisiert (siehe \autoref{msg:update_subscription}).
	
	\item \textbf{Umgebungsmodell deabonnieren [Vorbedingung: bereits abonniert]} \\
	Ein Fahrzeug kann das Umgebungsmodell deabonnieren, woraufhin keine neuen Modelle mehr vom Server an das Fahrzeug übermittelt werden sollen.
	Ein Abonnement wird durch eine \textit{UpdateSubscription}-Nachricht aktualisiert (siehe \autoref{msg:update_subscription}).
	
	
	\item \textbf{Als Sensor registrieren [Vorbedingung: noch nicht registriert]} \\
	Ein neu verbundener Sensor kann sich dem Server gegenüber als Sensor registrieren.
	Eine Registrierung kann für jede Verbindung nur einmal vorgenommen werden und wird durch die Übermittlung einer \textit{ClientRegistration}-Nachricht durchgeführt (siehe \autoref{msg:client_registration}).
	
	\item \textbf{SensorFrame zusenden [Vorbedingung: als Sensor registriert]} \\
	Ein Sensor kann dem Server eine \textit{SensorFrame}-Nachricht übermitteln.
	Der Server soll diese Nachricht dem Fusions-Algorithmus weiterleiten.
	
	\item \textbf{SensorIdleFrame zusenden [Vorbedingung: als Sensor registriert]} \\
	Ein Sensor kann dem Server eine \textit{SensorIdleFrame}-Nachricht übermitteln.
	
\end{itemize}

\subsection{Sensor}

In \autoref{use_case:server_sensor} sind Anwendungsfälle aufgezeigt, die durch den Sensor dem MEC-Server angeboten werden.


\begin{figure}[H]
	\begin{tikzpicture}[node distance=5]
		\begin{umlsystem}{Sensor}
			\umlusecase[y=0,name=u1]{deabonnieren}
			\umlusecase[y=2,name=u2]{abonnieren}
		\end{umlsystem} 
		\umlactor[x=-4.5,y=1]{MEC-Server}
		\umlassoc{MEC-Server}{u1}
		\umlassoc{MEC-Server}{u2} 
	\end{tikzpicture}
	\centering
	\caption{Anwendungsfalldiagramm des Sensors}
	\label{use_case:server_sensor}
\end{figure}

\begin{itemize}
	\item \textbf{Abonnieren [Vorbedingung: Sensor hat sich als Sensor registriert]} \\
	Der MEC-Server kann einen Sensor abonnieren um \textit{SensorFrame}-Nachrichten (siehe \autoref{msg:sensor_frame}) zu erhalten.
	Ein Abonnement wird durch eine \textit{UpdateSubscription}-Nachricht aktualisiert (siehe \autoref{msg:update_subscription}).
	
	\item \textbf{Deabonnieren [Vorbedingung: Sensor hat sich als Sensor registriert]} \\
	Der MEC-Server kann ein Abonnement aufkündigen um keine weiteren \textit{SensorFrame}-Nachrichten (siehe \autoref{msg:sensor_frame}) zu erhalten.
	Ein Abonnement wird durch eine \textit{UpdateSubscription}-Nachricht aktualisiert (siehe \autoref{msg:update_subscription}).
	
\end{itemize}


\section{Fahrzeug}

In \autoref{use_case:server_vehicle} sind Anwendungsfälle aufgezeigt, die durch das Fahrzeug dem MEC-Server angeboten werden.

\begin{figure}[H]
	\begin{tikzpicture}[node distance=5]
		\begin{umlsystem}{Fahrzeug}
			\umlusecase[y=0,name=u1]{Umgebungsmodell zusenden}
			\umlusecase[y=2,name=u2]{Sektoren zusenden}
		\end{umlsystem} 
		\umlactor[x=-6,y=1]{MEC-Server}
		\umlassoc{MEC-Server}{u1}
		\umlassoc{MEC-Server}{u2}
	\end{tikzpicture}
	\centering
	\caption{Anwendungsfalldiagramm des Fahrzeugs}
	\label{use_case:server_vehicle}
\end{figure}

\begin{itemize}
	\item \textbf{Sektoren zusenden [Vorbedingung: Fahrzeug ist registriert, einmalig]} \\
	Der MEC-Server soll dem Fahrzeug einmalig nach Registrierung alle bekannten Sektoren in einer \textit{InitMessage}-Nachricht (siehe \autoref{msg:init_message}) zusenden.
	
	\item \textbf{Umgebungsmodell zusenden [Vorbedingung: Fahrzeug hat Abonnement]} \\
	Der MEC-Server kann dem Fahrzeug Umgebungsmodelle mit \textit{EnvironmentFrame}-Nachrichten (siehe \autoref{msg:environment_frame}) zusenden.
	
\end{itemize}

%\begin{figure}[H]
%	\begin{tikzpicture}[node distance=5]
%		\begin{umlsystem}{Fusions-Algorithmus}
%			\umlusecase[y=0,name=u1]{Sensordaten übergeben}
%		\end{umlsystem} 
%		\umlactor[x=-5,y=0]{MEC-Server}
%		\umlassoc{MEC-Server}{u1}
%	\end{tikzpicture}
%	\centering
%	\label{use_case:server_algorithmus}
%	\caption{Use Case Diagramm des MEC-Servers gegenüber dem Fusions-Algorithmus}
%\end{figure}


\newpage
\section{Nachrichtenanalyse}
\label{analysis:messages}

\begin{wrapfigure}{R}[-1.5em]{.5\textwidth}
	\centering
	\begin{bytefield}[bitwidth=.45em,bitheight=.7em]{32}
		\bitheader{0,31} \\
		
		\begin{rightwordgroup}{Kopf}
			\wordbox{4}{ASN.1 Nachrichtenlänge \textbf{$n$}} \\
			\wordbox{4}{ASN.1 Nachrichtentyp}
		\end{rightwordgroup} \\
		
		\begin{rightwordgroup}{Länge in\\\textbf{$n$} Bytes}
			\wordbox[lrt]{8}{ASN.1 Nachricht} \\
			\skippedwords \\
			\wordbox[lrb]{2}{}
		\end{rightwordgroup}
	\end{bytefield}
	\caption{ASN.1 Nachricht mit Kopfdaten}
	\label{message:structure}
\end{wrapfigure}

Eine ASN.1 Nachricht wird, wie in \autoref{message:structure} zu sehen, mit  vorangestellten Kopfdaten versendet.
Diese Kopfdaten enthalten die Länge der ASN.1 Nachricht und dessen Typ, der bei der Dekodierung beim Empfängers bekannt sein muss.
Die Nachrichtenlänge und der Nachrichtentyp werden als 32-Bit lange und vorzeichenlose Ganzzahlen übermittelt und entsprechen damit dem Datentyp \rustcinline{u32} in Rust.
Sie werden als \enquote{Big-Endian} auf dem Datenstrom dargestellt.

Die Nachrichtenlänge gibt die Länge der ASN.1 Nachricht in Bytes an.

Für den Nachrichtentyp sind nur die in \autoref{message:types} gelisteten Werte gültig und werden im Anschluss erklärt.

\begin{figure}[H]
	\centering
	\begin{tabular}{r|l|l}
		Typ & Nachricht & Aus Sichtweise des Servers \\
		\hline
		0 & \enquote{Nicht definiert} & -- \\
		1 & ClientRegistration & eingehend \\
		2 & SensorFrame & eingehend \\
		3 & EnvironmentFrame & ausgehend \\
		4 & UpdateSubscription & bidirektional \\
		5 & InitMessage & ausgehend \\
		6 & RoadClearanceFrame & ausgehend \\
		7 & SensorIdleFrame & eingehend \\
		8 & UpdateStatus & \todo{eingehend} \\
	\end{tabular}
	\caption{Gültige ASN.1 Nachrichtentypen}
	\label{message:types}
\end{figure}

\subsection{ClientRegistration}
\label{msg:client_registration}

\todo{Woher kommt die? Anforderungen? Use Cases? Kontextdiagramm?}

Die ClientRegistration-Nachricht sendet der Client nach dem Verbindungsaufbau dem Server zu, um mitzuteilen, ob es sich um einen Sensor oder ein Fahrzeug handelt.

\subsection{SensorFrame}
\label{msg:sensor_frame}

Die SensorFrame-Nachricht wird vom Sensor an den Server versendet und beschreibt Objekte, die der Sensor erkannt hat.

\subsection{EnvironmentFrame}
\label{msg:environment_frame}

Die EnvironmentFrame-Nachricht wird vom Server an das Fahrzeug versendet und enthält das Ergebnis des Fusions-Algorithmus.

\subsection{UpdateSubscription}
\label{msg:update_subscription}

Die UpdateSubscription-Nachricht wird vom Server an den Sensor oder vom Fahrzeug an den Server gesendet.
\todo{enthält subscribe oder unsubscribe}
\todo{hier falsch: Der Sensor versendet bei einer Anmeldung erkannte Objekte (siehe \autoref{msg:sensor_frame}), während der Server dem Fahrzeug die Ergebnisse aus dem Fusions-Algorithmus (siehe \autoref{msg:environment_frame}) zusendet.
Bei einer Abmeldung wird das Versenden der Nachrichten eingestellt.}

\subsection{InitMessage}
\label{msg:init_message}

Die InitMessage-Nachricht wird vom Server, nach einer Fahrzeuganmeldung, an das Fahrzeug gesendet und beinhaltet Koordinaten über die Sektoren, die die Sensoren beobachten.

\subsection{RoadClearanceFrame}

Die RoadClearanceFrame-Nachricht wird vom Server an das Fahrzeug versendet und kann Verkehrs-, Wetter- und andere Informationen über die Sektoren enthalten.

\subsection{SensorIdleFrame}

Die SensorIdleFrame-Nachricht wird vom Sensor in regelmäßigen Zeitintervallen an den Server versendet, wenn der Sensor nicht abonniert ist.
Die Anwesenheit des Sensors wird somit festgestellt.

\subsection{UpdateStatus}

\todo{huh?}


%\begin{figure}[H]
%	\centering
%	\begin{bytefield}[bitwidth=.25em,bitheight=2em]{128}
%		\bitheader{0,32,64} \\
%		
%		\bitbox{32}{Länge $n$}
%		\bitbox{32}{Nachricht Id}
%		\bitbox{64}{ASN.1 Nachricht der Länge $n$}
%		%		\skippedbits
%		%		\bitbox[tbr]{32}{}
%	\end{bytefield}
%	\caption{Aufbau eines WebSocket-Frames \cite[28]{ieft:websockets}}
%	\label{figure:websockets:frame}
%\end{figure}
%\todo{alternativ:}
%\begin{figure}[H]
%	\centering
%	\begin{tikzpicture}
%		\draw[green!2,fill] (0,0) rectangle (2,-3);
%		\draw[red!6,fill] (2,0) rectangle (6,-3);
%		\draw[green!8,fill] (6,0) rectangle (14,-3);
%		\draw[red!2,fill] (14,0) rectangle (16,-3);
%		
%		\draw[gray!40, very thin] (0,0) -- (16,0);
%		\draw[gray!40, very thin] (0,-2) -- (16,-2);
%		\draw (2,0) -- (14,0);
%		\draw (2,-2) -- (14,-2);
%		\draw[gray!40, very thin] (2,-3) -- (2,0.2);
%		\draw (2,0.5) node {0};
%		\draw (3,-1) node {Länge $n$};
%%		\draw (3,-1.7) node {\rustcinline{u32}};
%		\draw[gray!40, very thin] (4,-2) -- (4,0.2);
%		\draw (4,0.5) node {32};
%		\draw (5,-1) node {Id};
%%		\draw (5,-1.7) node {\rustcinline{u32}};
%		\draw[gray!40, very thin] (6,-3) -- (6,0.2);
%		\draw (6,0.5) node {64};
%		\draw[gray!40, very thin] (14,-3) -- (14,0.2);
%		\draw (14,0.5) node {$64 + n$};
%		\draw (10,-1) node {ASN.1 Nachricht};
%		
%		\draw (4,-2.5) node {Kopf};
%		\draw (10,-2.5) node {Korpus};
%	\end{tikzpicture}
%	\label{message:structure_alt}
%	\caption{Aufbau einer Nachricht auf dem Datenstrom}
%\end{figure}


\section{Schnittstellenanalyse}


\begin{figure}[H]
	\begin{tikzpicture}
		\begin{umlseqdiag} 
			\umlactor[x=0,no ddots]{Sensor}
			\umlactor[x=11,no ddots]{MEC-Server}
			\begin{umlcall}[dt=4, op={ClientRegistration\{type = sensor, ..\}}, type=asynchron]{Sensor}{MEC-Server}
				\begin{umlcall}[dt=5, op={UpdateSubscription\{subscription-status = unsubscribe, ..\}}, type=asynchron]{MEC-Server}{Sensor}
				\end{umlcall}
			\end{umlcall}
			
			\begin{umlfragment}[type=loop]
				\begin{umlcall}[dt=7, op={SensorIdleFrame\{..\}}, type=asynchron]{Sensor}{MEC-Server}
				\end{umlcall}
			\end{umlfragment}
			
			\begin{umlcall}[dt=7, op={UpdateSubscription\{subscription-status = subscribe, ..\}}, type=asynchron]{MEC-Server}{Sensor}
			\end{umlcall}
			
			\begin{umlfragment}[type=loop]
				\begin{umlcall}[dt=7, op={SensorFrame\{..\}}, type=asynchron]{Sensor}{MEC-Server}
				\end{umlcall}
			\end{umlfragment}
			
			\begin{umlcall}[dt=7, op={UpdateSubscription\{subscription-status = unsubscribe, ..\}}, type=asynchron]{MEC-Server}{Sensor}
			\end{umlcall}
			
			\begin{umlfragment}[type=loop]
				\begin{umlcall}[dt=7, op={SensorIdleFrame\{..\}}, type=asynchron]{Sensor}{MEC-Server}
				\end{umlcall}
			\end{umlfragment}
		\end{umlseqdiag}
	\end{tikzpicture}
	\centering
	\label{seq_dia:sensor}
	\caption{Kommunikation zwischen Sensor und MEC-Server}
\end{figure}


\begin{figure}[H]
	\begin{tikzpicture}
		\begin{umlseqdiag} 
			\umlactor[x=0,no ddots]{Fahrzeug}
			\umlactor[x=11,no ddots]{MEC-Server}
			\begin{umlcall}[dt=4, op={ClientRegistration\{type = vehicle, ..\}}, type=asynchron]{Fahrzeug}{MEC-Server}
				\begin{umlcall}[dt=5, op={InitMessage\{..\}}, type=asynchron]{MEC-Server}{Fahrzeug}
				\end{umlcall}
			\end{umlcall}
			\begin{umlcall}[dt=5, op={UpdateSubscription\{subscription-status = subscribe, ..\}}, type=asynchron]{Fahrzeug}{MEC-Server}
			\end{umlcall}
			\begin{umlfragment}[type=loop] 
				\begin{umlcall}[dt=8, op={EnvironmentFrame\{..\}}, type=asynchron]{MEC-Server}{Fahrzeug}
				\end{umlcall}
			\end{umlfragment}
			
			\begin{umlcall}[dt=6, op={UpdateSubscription\{subscription-status = unsubscribe, ..\}}, type=asynchron]{Fahrzeug}{MEC-Server}
			\end{umlcall}
		\end{umlseqdiag}
	\end{tikzpicture}
	\centering
	\label{seq_dia:vehicle}
	\caption{Kommunikation zwischen Fahrzeug und MEC-Server}
\end{figure}


\begin{figure}[H]
	\begin{tikzpicture}
		\begin{umlseqdiag} 
			\umlactor[x=0,no ddots]{Sensor}
			\umlactor[x=6,no ddots]{MEC-Server}
			\umlactor[x=12,no ddots]{Fahrzeug}
			
			
			\begin{umlcall}[dt=5, op={register}, type=asynchron]{Sensor}{MEC-Server}
				\begin{umlcall}[dt=5, op={unsubscribe}, type=asynchron]{MEC-Server}{Sensor}
				\end{umlcall}
			\end{umlcall}
			
			
			\begin{umlfragment}[type=loop] 
				\begin{umlcall}[dt=6,op={SensorIdleFrame}, type=asynchron]{Sensor}{MEC-Server}
				\end{umlcall}
			\end{umlfragment}
			
			\begin{umlcall}[dt=28,op={register}, type=asynchron]{Fahrzeug}{MEC-Server}
				\begin{umlcall}[dt=5,op={InitMessage}, type=asynchron]{MEC-Server}{Fahrzeug}
				\end{umlcall}
				\begin{umlcall}[dt=0,op={subscribe}, type=asynchron]{MEC-Server}{Sensor}
				\end{umlcall}
			\end{umlcall}
			
			\begin{umlfragment}[type=loop] 
				\begin{umlcall}[dt=8, op={SensorFrame}, type=asynchron]{Sensor}{MEC-Server}
				\end{umlcall}
			\end{umlfragment}
			
			\begin{umlcall}[dt=14,op={subscribe}, type=asynchron]{Fahrzeug}{MEC-Server}
			\end{umlcall}
			
			\begin{umlfragment}[type=loop] 
				\begin{umlcall}[dt=15,op={SensorFrame}, type=asynchron]{Sensor}{MEC-Server}
					\begin{umlcall}[dt=2, op={EnvironmentFrame}, type=asynchron]{MEC-Server}{Fahrzeug}
					\end{umlcall}
				\end{umlcall}
			\end{umlfragment}
			
			\begin{umlcall}[dt=8,op={unsubscribe}, type=asynchron]{Fahrzeug}{MEC-Server}
			\end{umlcall}
			
			
			\begin{umlfragment}[type=loop] 
				\begin{umlcall}[dt=15, op={SensorFrame}, type=asynchron]{Sensor}{MEC-Server}
				\end{umlcall}
			\end{umlfragment}
			
			\umlsdnode[dt=11]{Fahrzeug}
			\draw[dashed] (-2,-16.5) -- (14,-16.5);
			\draw (9.225,-16.25) node {Fahrzeug trennt die Verbindung};

			\begin{umlcall}[dt=8,op={unsubscribe}, type=asynchron]{MEC-Server}{Sensor}
			\end{umlcall}
			
			\begin{umlfragment}[type=loop] 
				\begin{umlcall}[dt=6,op={SensorIdleFrame}, type=asynchron]{Sensor}{MEC-Server}
				\end{umlcall}
			\end{umlfragment}
		\end{umlseqdiag}
	\end{tikzpicture}
	\centering
	\label{seq_dia:both}
	\caption{Interaktion des MEC-Servers mit einem Fahrzeug und einem Sensor}
\end{figure}
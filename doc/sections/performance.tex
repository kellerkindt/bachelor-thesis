
%\chapter{Hochperformante, serverbasierte Kommunikationsplattform}
\chapter{\todo{\enquote{Systemrelevant}?}}
\label{com_plattform}

Dieses Kapitel erläutert weitere, relevante Themen die zur Umsetzung der \enquote{hochperformante, serverbasierte Kommunikationsplattform} wichtig sind.

\section{Echtzeitsysteme}
\label{real_time_systems}
Echtzeitsysteme zeichnen sich im allgemeinen dadurch aus, eine Aufgabe in einem zuvor vorgegebenen Zeitraum bearbeiten zu können.
Es existiert zu einer Aufgabe also immer eine Frist.
Bei der Bewertung der Richtigkeit eines Systems, wird die Fähigkeit, eine Frist einhalten zu können, auch bewertet \cite[2]{perf:buttazzo2006soft}.
Je nach Art des Echtzeitsystems, wird diese First jedoch unterschiedlich gewichtet:

\begin{itemize}
	\item Bei einem harten Echtzeitsystem kann eine Überschreitung der Frist einen katastrophalen Ausgang haben.
	Selbst im schlimmsten Fall darf diese Frist nicht überschritten werden.
	Deswegen wird in einem harten Echtzeitsystem die im maximale Reaktionszeit dem Zeitraum bis zur Frist gegenübergestellt \cite[75]{douglass2003real}.
	Ein Ergebnis nach Ablauf der Frist wird als nutzlos gewertet \cite[2]{perf:wang2017real}.
	
	Zum Beispiel könnte eine zu späte Auswertung von Beschleunigungsdaten in einem Flugzeug zu einer verzögerten und mittlerweile falschen Reaktion und daraufhin zu einem Absturz führen \cite[5]{perf:laplante2004real}.
	
	\item Bei einem weichen Echtzeitsystem resultiert die Überschreitung des vorgegebenes Zeitraums nicht in eine Katastrophe.
	Es wird die durchschnittliche Reaktionszeit dem Zeitraum bis zur Frist gegenübergestellt, eine seltene und unter last auftretende Überschreitung wird in kauf genommen \cite[76]{douglass2003real}.
	Das System führt in so einem Fall weiterhin seine Aufgaben aus, die Performance wird aber \todo{abgewertet} eingestuft.
%	Weiche Echtzeitsysteme können sogar überhaupt keine Frist haben, sondern die Aufgabe, die Antwortzeit so gering wie möglich zu gehalten \cite[4]{perf:buttazzo2006soft}.
\end{itemize}

\section{Mobile Edge Computing}

\todo{refactor text}

Mobile Edge Computing (\gls{mec}) bezeichnet Recheneinheiten, die eine Cloudähnliche Umgebung am Rande des Mobilfunknetzes schaffen \cite[4]{etsi:mec}.
Wenn in dieser Arbeit auf MEC Bezug genommen wird, sind damit explizit direkt an Funkmasten montierte Recheneinheiten gemeint.
Dadurch, dass die Recheneinheiten direkt an eine Antenne angeschlossen sind, können sie Anfragen aus dem Abdeckungsbereich der Antenne deutlich schneller beantworten (Latenz kleiner 20ms) als Cloudlösungen (Latenz ca 100ms) \cite[2]{perf:mec:fraunhofer}.
Hierfür werden die Anfragen aus dem Mobilfunknetz direkt an die Recheneinheit weitergeleitet, anstatt über einen Provider eine Internetverbindung zu einer Cloudlösung aufzubauen.
In dem \gls{mec}-View Projekt wird eine VM in einer \gls{mec} Recheneinheit ausgeführt und die Verbindungen zu den Sensoren und dem Fahrzeuge \todo{durchgeschleust}.


%\section{Architekturmuster? oder erst während der Implementation?}
%\subsection{Was ist dann ein hochperformantes System}
%\section{Serverbasierte Kommunikationsplattform: MEC}

		

\section{ASN.1}

\begin{quotation}
	\textit{\enquote{
			ASN.1 has a long record of accomplishment, having been in use since 1984. It has evolved over time to meet industry needs, such as PER support for the bandwidth-constrained wireless industry and XML support for easy use of common Web browsers.
		}}
	\cite{asn:itu:asn.1}
\end{quotation}

Die Notationsform \gls{asn} ermöglicht abstrakte Datentypen und Wertebereich zu beschreiben \cite{asn:layman}.
Die Beschreibungen können anschließend zu Quellcode einer theoretisch\footnote{Es gibt keine Einschränkungen seitens des Standards, aber entsprechende Compiler zu finden erweist sich als schwierig \todo{ref impl Schwierigkeiten mit ASN+Rust}} beliebigen Programmiersprache compiliert werden.
Beschriebene Datentypen werden dadurch als native Konstrukte dargestellt und können mittels einer der standardisierten (oder auch eigenen \cite{asn:itu:ecn}) Encodierungen serialisiert werden.

Um den Austausch zwischen verschiedenen Anwendungen und Systemen zu ermöglichen, sind durch die \gls{itu}  bereits einige Encodierungen standardisiert \cite[8]{asn:itu:x691}.
Für diese Arbeit ist aber einzig der PER bzw. uPER Standard relevant, da der Server diese Encodierung verwenden muss, um mit den Sensoren und den Autos zu kommunizieren (Anforderung in \autoref{req:com:asn}).

Andere, bekanntere Verfahren werden hier nur kurz erwähnt:
\begin{itemize}
	\item \textbf{BER} (\underline{B}asic \underline{E}ncoding \underline{R}ules): Flexible binäre Encodierung \cite{asn:wiki:x690}, spezifiziert in X.690 \cite{asn:itu:x690} und ISO/IEC 8825-1 \cite{asn:iso}.
	\item \textbf{CER} (\underline{C}anonical \underline{E}ncoding \underline{R}ules): Reduziert BER durch die Restriktion, die Enden von Datenfelder speziell zu Markieren anstatt deren Größe zu übermitteln, eignet sich gut für große Nachrichten \cite{asn:wiki:x690}, spezifiziert in X.690 \cite{asn:itu:x690} und ISO/IEC 8825-1 \cite{asn:iso}.
	\item \textbf{DER} (\underline{D}istinguished \underline{E}ncoding \underline{R}ules): Reduziert BER durch die Restriktion Größeninformationen zu Datenfeldern in den Metadaten zu übermitteln, eignet sich gut für kleine Nachrichten \cite{asn:wiki:x690}, spezifiziert in X.690 \cite{asn:itu:x690} und ISO/IEC 8825-1 \cite{asn:iso}.
	\item \textbf{XER} (\underline{X}ML \underline{E}ncoding \underline{R}ules): Beschreibt den Wechsel der Darstellung zwischen ASN.1 und XML, spezifiziert in X.693 \cite{asn:itu:x693} und ISO/IEC 8825-4 \cite{asn:iso}.
\end{itemize}

\subsubsection{PER und UPER}

Die \underline{P}acked \underline{E}ncoding \underline{R}ule ist in in X.691 \cite{asn:itu:x691} und ISO/IEC 8825-2 \cite{asn:iso} spezifiziert.
Sie beschreibt eine Encodierung, die Daten kompakt -- also in wenigen Bytes -- serialisiert.
Zu PER sind mehrere Variationen spezifiziert, für diese Arbeit ist jedoch nur UPER (unaligned PER) von Bedeutung.
Im Gegensatz zu anderen Variationen bestehen Datenbausteine in UPER nicht aus ganzen Bytes, sondern aus unterschiedlich vielen Bits.
Eine serialisierte Nachricht ist deswegen nicht N-Bytes sondern N-Bits lang.
An den resultierenden Bitstring dürfen 0-Bits angehängt werden, um diesen als Bytestring wandeln zu können.
Durch dieses Verfahren ist die Nachricht noch kürzer darstellbar.

\todo{sources:}
Für den Einsatz bei Funkverbindungen ist diese Encodierung von Vorteil, da bei der Übermittlung einer Nachricht kein anderen Kommunikationsteilnehmer auf dieser Frequenz etwas übermitteln kann.
Eine kürzere Nachricht blockiert eine Frequenz kürzer, weshalb kürzere Nachrichten einen höheren Durchsatz erlauben.
Im Mobilfunkbereich ist dies von besonderer Bedeutung, da das Medium von vielen Teilnehmern gleichzeitig und über eine große Fläche geteilt wird. \todo{michael.refactor\_this\_shit()}

	
\section{Test-Driven Development}
\label{tdd}

\begin{quotation}
	\textit{\enquote{Failure is progress.}}
	\cite[5]{tdd}
\end{quotation}
%\begin{quotation}
%	\textit{\enquote{Make it run, make it good.}}
%	\cite[24]{tdd}
%\end{quotation}

Bei der Test-getriebenen Entwicklung werden Tests in den Vordergrund gestellt.
Die Implementierung einer neuen Funktionalität wird durch neue Tests, die die Anforderung repräsentieren, eingeleitet.
Erst nachdem ein Test erfolgreich feststellt, dass die geforderte Funktionalität noch nicht vorhanden ist, wird mit der Implementierung begonnen.
Eine schnelle Implementierung hat hierbei die höchste Priorität und erlaubt temporär auch eine limitierende, stinkende und naive Vorgehensweise \cite[7]{tdd}.
%Der neue Code, der den Test zufriedenstellt, darf hierbei kurzzeitig \enquote{stinken} \todo{cite} und Qualitätsstandards verletzten.
Direkt im Anschluss wird ein Refactoring\footnote{Verbesserung des Codes und der Struktur ohne Änderung der Funktionalität} durchgeführt, um die Qualitätsstandards wieder einzuhalten.
Diese drei Phasen werden \enquote{red/green/refactor} bezeichnet:

\begin{itemize}
	\item \textbf{red}: Ein neuer Test wird erstellt, dieser stellt erfolgreich die Abwesenheit der Funktionalität fest, eine rote Fehlermeldung ist zu sehen.
	\item \textbf{green}: Der Test wird durch neuen Code zufriedengestellt; eine positive Ausgabe bestätigt dies. Eine schnelle Implementierung wird hierbei temporär einer hochwertigen bevorzugt \cite[24]{tdd}, da ein erfolgreicher Test das Selbstvertrauen beim Refactorn stärke und helfen würde, Fehlerhafte Tests zu finden \cite[152]{tdd}.
	\item \textbf{refactor}: Der neue Code wird aufgeräumt und verbessert um den Qualitätsstandards gerecht zu werden. Ein andauernder Testdurchlauf versichert dabei keine Regression der Funktionalität.
\end{itemize}

Die Testgröße und der daraus resultierende Umfang der neuen Funktionalität, wird durch die Zuversichtlichkeit des Entwicklers gesteuert \cite[42]{tdd}.
Eine hohe Zuversicht führe zu größeren Tests, die etwas mehr Funktionalität auf einmal prüfen, während eine hohe Unsicherheit zu vielen kleinen Tests führen würden.
Daraus Resultiert, dass gerade komplexe Algorithmen mit vielen Tests abgesichert sein sollten.
Bestehende Tests bilden ein Sicherheitsnetz, mit dem Defekte Änderungen umgehend detektiert werden können.

%\todo{manchmal stellt man beim erstellen des tests fest, dass die Anforderung blöd ist}
Test-getriebene Entwicklung verändert auch die Vorgehensweise bei der Implementation von Anforderungen. Anstatt zu fragen \enquote{Wie würde ich das Implementieren?}, wird überlegt \enquote{Wie würde ich das Testen?} \cite[39]{tdd}, womit auch implizit gefragt wird, wie die äußeren Schnittstelle idealerweise aussehen sollen \cite[4]{tdd}.


%\todo{p.11, shift of practice in p.13, \enquote{Make it run, make it good} p.24, no changes unless enough motivation p.34, tests run -> not broken anything p.37, never interrupt an interruption p.41, TDD is a steering process, unsure -> smaller steps, sure -> bigger steps p.42, Test First can reduce stress p.127}

%\todo{mention stress reduction?}


\section{Funktionale Sicherheit}
\label{com:safety}

\enquote{Sicherheit} ist im Deutschen kein eindeutiger Begriff.
Sowohl \enquote{Sichersein vor Gefahr oder Schaden} (\textit{to be safe}), \enquote{Freisein von Fehlern oder Irrtümern} (\textit{to be confident}) oder \enquote{Schutz vor Gefahren, die von außen auf Systeme oder Personen einwirken} (\textit{security}) könnten mit \enquote{sicher sein} gemeint sein \cite[5-6]{safety}.
Deswegen ist es wichtig, den Begriff \enquote{funktionale Sicherheit} kurz zu ergründen.

Bei funktionaler Sicherheit (\textit{safety}) geht es um die Betriebssicherheit, eine \enquote{Freiheit von unvertretbaren Risiken} \cite[6]{safety}.
Unvertretbare Risiken sind in erster Linie Personenschäden, weswegen einheitliche Regularien in Normen wie der IEC 61508 bzw der DIN EN 61508 festgehalten sind.
Für den Automobilbereich wurde die Norm in der ISO 26262 angepasst, um u.a. eine Einzelabnahme eines jeden Fahrzeuges, durch eine Gesamtabnahme des Produktes zu ermöglichen \cite[14]{safety}.
\todo{Erwähnung?: Einstufung (A)SIL -> Gegenmaßnahmen}

Durch die Test-getriebene Entwicklung und der Verwendung von Rust und dessen Garantien (siehe \autoref{rust:guarantees}), sollen Fehler reduziert und eine möglichst sichere Implementierung geschaffen werden.
Eine Entwicklung nach ISO 26262 findet nicht statt, da dies zum Einen nicht durch das Forschungsprojekt gefordert ist und zum Anderen den Umfang dieser Bachelorarbeit überschreitet.

%\todo{such shiat argument, arguemnt: more rust -> f. safety}
%Obwohl funktionale Sicherheit für das Forschungsprojekt \gls{mec}-View eine nicht irrelevante Rolle Bedeutung hat, wird in dieser Bachelorarbeit keine Entwicklung nach ISO 26262 vorgenommen.
%Zum einen sollen die Garantien von Rust (siehe \autoref{rust:guarantees}) viele mögliche Fehlerquellen generell ausschließen, zum anderen soll durch die Test-getriebene Entwicklung (siehe \autoref{tdd}) die Fehlerwahrscheinlichkeit weiter reduziert werden.
%Zuletzt ist anzumerken, dass das System nicht für einen Endanwender konzipiert ist, sondern nur durch entsprechendes Fachpersonal betrieben und in Notfallsituationen eingegriffen wird.

	
%\section{Sensordaten?}
	
%\section{TCP?}
	
%\todo{Kommunikation als Socket, FiFo, Fehlerkorrektur, erneutes senden bei Fehlern, richtige Reiehenfolge...}

\chapter{Hochperformante, serverbasierte Kommunikationsplattform}
\label{com_plattform}

Dieses Kapitel erläutert den Begriff \enquote{hochperformante, serverbasierte Kommunikationsplattform} und vermittelt Basiswissen hierzu.

\section{Echtzeitsysteme}
Echtzeitsysteme zeichnen sich im allgemeinen dadurch aus, eine Aufgabe in einem zuvor vorgegebenen Zeitraum bearbeiten zu können.
Es existiert zu einer Aufgabe also immer eine Frist.
Bei der Bewertung der Richtigkeit eines Systems, wird die Fähigkeit, eine Frist einhalten zu können, auch bewertet \cite[2]{perf:buttazzo2006soft}.
Je nach Art des Echtzeitsystems, wird diese First jedoch unterschiedlich gewichtet:

\begin{itemize}
	\item Bei einem harten Echtzeitsystem kann eine Überschreitung der Frist einen katastrophalen Ausgang haben.
	Selbst im schlimmsten Fall darf diese Frist nicht überschritten werden.
	Deswegen wird in einem harten Echtzeitsystem die im maximale Reaktionszeit dem Zeitraum bis zur Frist gegenübergestellt \cite[75]{douglass2003real}.
	Ein Ergebnis nach Ablauf der Frist wird als nutzlos gewertet \cite[2]{perf:wang2017real}.
	
	Zum Beispiel könnte eine zu späte Auswertung von Beschleunigungsdaten in einem Flugzeug zu einer verzögerten und mittlerweile falschen Reaktion und daraufhin zu einem Absturz führen \cite[5]{perf:laplante2004real}.
	
	\item Bei einem weichen Echtzeitsystem resultiert die Überschreitung des vorgegebenes Zeitraums nicht in eine Katastrophe.
	Es wird die durchschnittliche Reaktionszeit dem Zeitraum bis zur Frist gegenübergestellt, eine seltene und unter last auftretende Überschreitung wird in kauf genommen \cite[76]{douglass2003real}.
	Das System führt in so einem Fall weiterhin seine Aufgaben aus, die Performance wird aber \todo{abgewertet} eingestuft.
	Weiche Echtzeitsysteme können sogar überhaupt keine Frist haben, sondern die Aufgabe, die Antwortzeit so gering wie möglich zu gehalten \cite[4]{perf:buttazzo2006soft}.
\end{itemize}

\subsection{Echtzeitnah}
	\label{com:near_real_time}
	\todo{?}
			
		\section{Funktionale Sicherheit}
			\label{com:safety}
			
		\subsection{Was ist dann ein hochperformantes System}
		\subsection{Low-Latency + Entwurfsmuster + Patterns? + Algorithmen?}
			\todo{Hochperformant -> parallel?}
			
			\todo{Design Pattern, Gamma et al, four important aspects}
			
			\todo{Real Time Design Patterns Buch: Ab Seite 141, verschiedene Systempatterns, microkernel \cite[151]{douglass2003real}? channel architektur pattern \cite[167]{douglass2003real}?}
			
			\todo{Message Queuing Pattern \cite[207]{douglass2003real}}
			
			\todo{Clean Architecture / Clean Code}
			
	\section{Serverbasierte Kommunikationsplattform: MEC}
	
		
		
	
\section{ASN.1}

\begin{quotation}
	\textit{\enquote{
			ASN.1 has a long record of accomplishment, having been in use since 1984. It has evolved over time to meet industry needs, such as PER support for the bandwidth-constrained wireless industry and XML support for easy use of common Web browsers.
		}}
	\cite{asn:itu:asn.1}
\end{quotation}

Die Notationsform \gls{asn} ermöglicht abstrakte Datentypen und Wertebereich zu beschreiben \cite{asn:layman}.
Die Beschreibungen können anschließend zu Quellcode einer theoretisch\footnote{Es gibt keine Einschränkungen seitens des Standards, aber entsprechende Compiler zu finden erweist sich als schwierig \todo{ref impl Schwierigkeiten mit ASN+Rust}} beliebigen Programmiersprache compiliert werden.
Beschriebene Datentypen werden dadurch als native Konstrukte dargestellt und können mittels einer der standardisierten (oder auch eigenen \cite{asn:itu:ecn}) Encodierungen serialisiert werden.

Um den Austausch zwischen verschiedenen Anwendungen und Systemen zu ermöglichen, sind durch die \gls{itu}  bereits einige Encodierungen standardisiert \cite[8]{asn:itu:x691}.
Für diese Arbeit ist aber einzig der PER bzw. uPER Standard relevant, da der Server diese Encodierung verwenden muss, um mit den Sensoren und den Autos zu kommunizieren (Anforderung in \autoref{req:com:asn}).

Andere, bekanntere Verfahren werden hier nur kurz erwähnt:
\begin{itemize}
	\item \textbf{BER} (\underline{B}asic \underline{E}ncoding \underline{R}ules): Flexible binäre Encodierung \cite{asn:wiki:x690}, spezifiziert in X.690 \cite{asn:itu:x690} und ISO/IEC 8825-1 \cite{asn:iso}.
	\item \textbf{CER} (\underline{C}anonical \underline{E}ncoding \underline{R}ules): Reduziert BER durch die Restriktion, die Enden von Datenfelder speziell zu Markieren anstatt deren Größe zu übermitteln, eignet sich gut für große Nachrichten \cite{asn:wiki:x690}, spezifiziert in X.690 \cite{asn:itu:x690} und ISO/IEC 8825-1 \cite{asn:iso}.
	\item \textbf{DER} (\underline{D}istinguished \underline{E}ncoding \underline{R}ules): Reduziert BER durch die Restriktion Größeninformationen zu Datenfeldern in den Metadaten zu übermitteln, eignet sich gut für kleine Nachrichten \cite{asn:wiki:x690}, spezifiziert in X.690 \cite{asn:itu:x690} und ISO/IEC 8825-1 \cite{asn:iso}.
	\item \textbf{XER} (\underline{X}ML \underline{E}ncoding \underline{R}ules): Beschreibt den Wechsel der Darstellung zwischen ASN.1 und XML, spezifiziert in X.693 \cite{asn:itu:x693} und ISO/IEC 8825-4 \cite{asn:iso}.
\end{itemize}

\subsubsection{PER und UPER}

Die \underline{P}acked \underline{E}ncoding \underline{R}ule ist in in X.691 \cite{asn:itu:x691} und ISO/IEC 8825-2 \cite{asn:iso} spezifiziert.
Sie beschreibt eine Encodierung, die Daten kompakt -- also in wenigen Bytes -- serialisiert.
Zu PER sind mehrere Variationen spezifiziert, für diese Arbeit ist jedoch nur UPER (unaligned PER) von Bedeutung.
Im Gegensatz zu anderen Variationen bestehen Datenbausteine in UPER nicht aus ganzen Bytes, sondern aus unterschiedlich vielen Bits.
Eine serialisierte Nachricht ist deswegen nicht N-Bytes sondern N-Bits lang.
An den resultierenden Bitstring dürfen 0-Bits angehängt werden, um diesen als Bytestring wandeln zu können.
Durch dieses Verfahren ist die Nachricht noch kürzer darstellbar.

\todo{sources:}
Für den Einsatz bei Funkverbindungen ist diese Encodierung von Vorteil, da bei der Übermittlung einer Nachricht kein anderen Kommunikationsteilnehmer auf dieser Frequenz etwas übermitteln kann.
Eine kürzere Nachricht blockiert eine Frequenz kürzer, weshalb kürzere Nachrichten einen höheren Durchsatz erlauben.
Im Mobilfunkbereich ist dies von besonderer Bedeutung, da das Medium von vielen Teilnehmern gleichzeitig und über eine große Fläche geteilt wird. \todo{michael.refactor\_this\_shit()}
	
	\section{Sensordaten?}
	
	\section{TCP?}
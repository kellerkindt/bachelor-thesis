\chapter{Zusammenfassung und Fazit}

\todo{.}
Rust wirbt mit vielen verlockenden Versprechen, sicher und auch performant Programmieren zu können.
Diese Versprechen un die gute Dokumentation haben mir geholfen, nach ersten Versuchen, die Programmiersprache nicht wieder auf die Seite zu legen.
Der sehr strenge und zu beginn geradezu verteufelte Compiler kann einem oft zum Augenbraun verziehen bringen.
Einige aus der objektorientierten Welt bekannte und verinnerlichten Prinzipien konnten oft nur in einer Abwandlung in Rust übertragen werden.
Herangehensweisen, die für einen selbst sicher erscheinen (bei der nebenläufigen Programmierung)  -- und es für diesen einen Verwendungszweck vermutlich sind -- lehnt der Compiler oft ab, da sie nicht als allgemein sicher bewiesen werden können.
Wenn man sich aber Zeit nimmt und versucht zu verstehen, was der Compiler grundsätzlich bemängelt, verliert der Compiler \todo{böse} und wird stattdessen langsam zum hilfsbereiten \todo{companion}.
Gerade weil nach einem erfolgreichem Compilevorgang, das Programm meist wie erwartet läuft.

Positiv zu erwähnen ist auch die Kontrolle, die Rust einem gegenüber zum Beispiel Java gibt.
Datenstrukturen

\todo{war rust toll?}

\todo{Eigentümerprinzip gewöhnungsbedürftig, learning curve, einem wird sehr deutlich unter die nase gehalt, wo etwas performance kostet}

\todo{messen ist nicht einfach}

\todo{setup, toolchain etc gut}

\todo{performance muss sich nicht verstecken, für funktional sichere Anwendungen interessant}

\todo{Paketmanager cargo, standardbibliothek}
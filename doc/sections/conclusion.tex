\chapter{Zusammenfassung und Fazit}

\todo{.}
Rust wirbt mit den verlockenden Versprechen, sicher und auch performant Programmieren zu können.
Diese Verlockungen in Kombination mit der guten Dokumentation waren nötig, um die Anfangszeit zu überwinden.
Gerade zu beginn, fühlt es sich an, als würde man gegen den Compiler kämpfen.
Änderungen im Quellcode mit direkt darauf folgenden erfolgreichen Compilevorgängen gleichen Wundern.
Rückblickend betrachtet, lag es aber vor allem an den aus der objektorientierten Welt bekannte und verinnerlichten Prinzipien, \todo{die einem das Programmieren erschwert haben}.

%Der sehr strenge und zu beginn geradezu verteufelte Compiler kann einem oft zum Augenbraun verziehen bringen.
Einige aus der objektorientierten Welt bekannte und verinnerlichten Prinzipien konnten oft nur in einer Abwandlung in Rust übertragen werden.
Herangehensweisen, die für einen selbst sicher erscheinen (bei der nebenläufigen Programmierung), lehnt der Compiler oft ab, da sie nicht als allgemein sicher bewiesen werden können.
Wenn man sich aber Zeit nimmt und versucht zu verstehen, was der Compiler grundsätzlich bemängelt, verliert der Compiler \todo{böse} und wird stattdessen langsam zum hilfsbereiten \todo{companion}.
Gerade weil nach einem erfolgreichem Compilevorgang, das Programm meist wie erwartet läuft.

Rust erzwingt fast die Trennung von Verantwortlichkeiten.


Positiv zu erwähnen ist auch die Kontrolle, die Rust einem gegenüber zum Beispiel Java gibt.
Datenstrukturen

Zu den positiven Überraschungen zählt die Umfangreiche Standardbibliothek und die immense Anzahl an Crates aus der öffentlichen \todo{repository}.
Zusammen mit Werkzeugen wie Cargo und Rustup ist die Installation und Pflege von Rust und die Verwendung von externen Crates \todo{berauschend} einfach und unkompliziert.

Letztendlich bleibt der Eindruck, dass die Programmiersprache Rust da ist um zu bleiben \todo{its here to stay}.
Weder im Laufzeitverhalten, Funktionsumfang oder Unterstützung stößt man auf Argumente, die gegen eine Verwendung von Rust sprechen würden.
Im Gegenteil, es macht sich das Gefühl breit, keine andere Programmiersprache könne die Anforderungen der funktionalen Sicherheit ausreichend erfüllen.

\todo{too much fanciness of rust?}


\todo{war rust toll?}

\todo{Eigentümerprinzip gewöhnungsbedürftig, learning curve, einem wird sehr deutlich unter die nase gehalt, wo etwas performance kostet}

\todo{messen ist nicht einfach}

\todo{setup, toolchain etc gut}

\todo{performance muss sich nicht verstecken, für funktional sichere Anwendungen interessant}

\todo{Paketmanager cargo, standardbibliothek}